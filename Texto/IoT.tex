\documentclass[
	% -- opções da classe memoir --
	article,			% indica que é um artigo acadêmico
	11pt,				% tamanho da fonte
	oneside,			% para impressão apenas no verso. Oposto a twoside
	a4paper,			% tamanho do papel. 
	% -- opções da classe abntex2 --
	%chapter=TITLE,		% títulos de capítulos convertidos em letras maiúsculas
	section=TITLE,		% títulos de seções convertidos em letras maiúsculas
	subsection=TITLE,	% títulos de subseções convertidos em letras maiúsculas
	%subsubsection=TITLE % títulos de subsubseções convertidos em letras maiúsculas
	% -- opções do pacote babel --
	english,			% idioma adicional para hifenização
	brazil,				% o último idioma é o principal do documento
	sumario=tradicional
	]{abntex2}


% ---
% PACOTES
% ---

% ---
% Pacotes fundamentais 
% ---
\usepackage{lmodern}			% Usa a fonte Latin Modern
\usepackage[T1]{fontenc}		% Selecao de codigos de fonte.
\usepackage[utf8]{inputenc}		% Codificacao do documento (conversão automática dos acentos)
\usepackage{indentfirst}		% Indenta o primeiro parágrafo de cada seção.
\usepackage{nomencl} 			% Lista de simbolos
\usepackage{color}				% Controle das cores
\usepackage{graphicx}			% Inclusão de gráficos
\usepackage{microtype} 			% para melhorias de justificação
% ---
		
% ---
% Pacotes adicionais, usados apenas no âmbito do Modelo Canônico do abnteX2
% ---
\usepackage{lipsum}				% para geração de dummy text
% ---
		
% ---
% Pacotes de citações
% ---
\usepackage[brazilian,hyperpageref]{backref}	 % Paginas com as citações na bibl
\usepackage[alf]{abntex2cite}	% Citações padrão ABNT
% ---

% ---
% Configurações do pacote backref
% Usado sem a opção hyperpageref de backref
\renewcommand{\backrefpagesname}{Citado na(s) página(s):~}
% Texto padrão antes do número das páginas
\renewcommand{\backref}{}
% Define os textos da citação
\renewcommand*{\backrefalt}[4]{
	\ifcase #1 %
		Nenhuma citação no texto.%
	\or
		Citado na página #2.%
	\else
		Citado #1 vezes nas páginas #2.%
	\fi}%
% ---

% ---
% Informações de dados para CAPA e FOLHA DE ROSTO
% ---
\titulo{Internet das Coisas}
\autor{Thiago Raulino Dal Pont}
\local{Brasil}
\data{XX de Janeiro de 2017}
% ---

% ---
% Configurações de aparência do PDF final

% alterando o aspecto da cor azul
\definecolor{blue}{RGB}{41,5,195}

% informações do PDF
\makeatletter
\hypersetup{
     	%pagebackref=true,
		pdftitle={\@title}, 
		pdfauthor={\@author},
    	pdfsubject={Modelo de artigo científico com abnTeX2},
	    pdfcreator={LaTeX with abnTeX2},
		pdfkeywords={abnt}{latex}{abntex}{abntex2}{atigo científico}, 
		colorlinks=true,       		% false: boxed links; true: colored links
    	linkcolor=blue,          	% color of internal links
    	citecolor=blue,        		% color of links to bibliography
    	filecolor=magenta,      		% color of file links
		urlcolor=blue,
		bookmarksdepth=4
}
\makeatother
% --- 

% ---
% compila o indice
% ---
\makeindex
% ---

% ---
% Altera as margens padrões
% ---
\setlrmarginsandblock{3cm}{2cm}{*}
\setulmarginsandblock{3cm}{2cm}{*}
\checkandfixthelayout
% ---

% --- 
% Espaçamentos entre linhas e parágrafos 
% --- 

% O tamanho do parágrafo é dado por:
\setlength{\parindent}{1.5cm}

% Controle do espaçamento entre um parágrafo e outro:
\setlength{\parskip}{0.0cm}  % tente também \onelineskip

% Espaçamento simples
\SingleSpacing

% ----
% Início do documento
% ----
\begin{document}

% Retira espaço extra obsoleto entre as frases.
\frenchspacing 

% ----------------------------------------------------------
% ELEMENTOS PRÉ-TEXTUAIS
% ----------------------------------------------------------

%---
%
% Se desejar escrever o artigo em duas colunas, descomente a linha abaixo
% e a linha com o texto ``FIM DE ARTIGO EM DUAS COLUNAS''. \twocolumn[    		% INICIO DE ARTIGO EM DUAS COLUNAS
%
%---
% página de titulo
\maketitle

% resumo em português
\begin{resumoumacoluna}
 \lipsum[1]
 
 \vspace{\onelineskip}
 
 \noindent
 \textbf{Palavras-chaves}: amplificador diferencial, amplificador de instrumentação.
\end{resumoumacoluna}

% ]  				% FIM DE ARTIGO EM DUAS COLUNAS
% ---

% ----------------------------------------------------------
% ELEMENTOS TEXTUAIS
% ----------------------------------------------------------
\textual

% ----------------------------------------------------------
% Introdução
% ----------------------------------------------------------
\section*{Introdução}
\addcontentsline{toc}{section}{Introdução}

%%%%%%%%%%%%%%%%%%%%%%%%%%%%%%%%%%%%%%%%%%%%%%%%%%%%%%%%%%%%%%%%%%%%







%%%%%%%%   PARTE 1   %%%%%%%%
\section{Conceito}


% TODO: INTRODUÇÃO

% Contextualização

% REVER
A tecnologia, com o passar dos anos, está cada vez mais presente nas indústrias, lares, comércios 
etc. ao mesmo tempo tornando-se indispensável para todas essas entidades. No entanto, nos últimos 
anos um novo paradigma está emergindo: a Internet das Coisas. A partir dela, a Internet vai deixar 
de existir como é vista hoje tornando, assim, onipresente.

% O que é 

O conceito de Internet das Coisas (IoT) está relacionado à interconexão de objetos distintos 
através de uma rede, sendo esta, muitas vezes, a Internet. Desse modo, elementos do mundo real, que 
antes funcionavam de maneira independente ao meio aos quais estavam inseridos, são capazes de 
interagir com outros objetos à sua volta e, assim, trocar informações que possam ser relevantes 
permitindo a agregação de novas funcionalidades.  Além disso, a IoT abre espaço para interação 
entre o mundo físico e o digital a partir de dispositivos capazes de capturar dados físicos no meio 
em que estão tais como, temperatura, distância etc., representá-los digitalmente e trasmití-los 
para outros dispositivos.

	
O termo ``Internet das Coisas'' foi citado pela primeira vez por Kevin Ashton, diretor executivo da 
AutoIDCentre do MIT, em 1999 enquanto realizava uma apresentação para promover a ideia do uso de 
Identificadores de Radio Frequência (RFID) na etiquetagem de produtos. O uso da tecnologia 
beneficiaria a logística da cadeia de produção \cite{kevin-ashton}. Apesar de o termo IoT ter sido 
usado apenas em 1999, aplicações práticas da ideia já existiam anos antes. Um exemplo disso, é a 
torradeira que podia ser ligada e desligada via internet criada em 1990 \cite{survey-suresh}.



% TODO: Projeções de grandes companhias (número de dispositivos, expansão)

A Internet das Coisas está em grande expansão. Estima-se que em 2020 cerca de 24 bilhões de 
dispositivos IoT estejam conectados, implicando em cerca de quatro dispositivos por pessoa. Para 
tanto, em torno de 6 trilhões de dólares serão investidos em desenvolvimento de tecnologias de 
hardware e software, como aplicações, segurança e dispositivos de hardware. Apesar da grande 
quantia investida, o setor é visto como promissor. Estima-se será gerado em torno de 13 trilhões de 
dólares 
em 2025 \cite{andrewmeola2016}. 

Para conectar uma grande quantidade de objetos são necessárias tecnologias, muitas delas sem fio, 
que permitam que os dispostivos interajam entre si trocando informações de maneira eficiente. A 
próxima seção tratará dessas tecnologias e a maneira com são organizadas para formar uma 
arquitetura. 

Falar sobre as interações (realidade aumentada e tudo mais) proporcionadas 
pelas tecnologias

% TODO: GANCHO PARA AS TECNOLOGIAS

% TODO: Empresas, associações que abraçaram a ideia

% TODO: Benefícios, facilidade comodidade

%%%%%%%%   PARTE 2   %%%%%%%%
\section{Tecnologias}

\subsection{Bluetooth}

% O que é
% MELHORAR e referenciar
O Bluetooth é uma especificação de rede WPAN, ou seja, rede sem-fio pessoal, sendo descrito e 
especificado pela IEEE 802.15.1. O Bluetooth foi criado na década de 90 com o objetivo de unir 
tecnologias distintas, tais como computadores, celulares entre outros (artigo de com. de dados). 
Além disso uma das principais características da tecnologia \textit{wireless} é o curto alcance de 
transmissão variando de centímetros até alguns metros (livro-programmers). 
% TODO: Referenciar o livro do programmers essentials.

A tecnologia vem sendo usada ao longo dos últimos anos em diversas aplicações como transferência de 
arquivos entre dispositivos, transmissão de áudio entre smartphones e fones sem fio, dispositivos 
capazes de determinar contexto, como os beacons, entre outros.

% Topologias
No IEEE 802.15.1 há suporte para criação de redes \textit{ad-hoc}, aos quais, é desnecessário uma 
infraestrutura de rede para conexão dos dispositivos. A partir disso é possível criar redes 
chamadas \textit{picorredes}, nas quais os dispositivos são organizados em até oito associados.

% Características
A tecnologia Bluetooth opera na faixa regulamentada de 2.4 GHz de uso livre em modo TDM com um 
delta de $625\mu$s, proporcionando uma taxa de transmissão máxima em torno de 2 Mb/s, podendo 
variar 
de acordo com o dispositivo e a categoria de tecnologia de Bluetooth utilizada. (Especificação v5).

% Forma de conexão.

\subsubsection{Categorias}

Segundo (especificação bluetooth), o Bluetooth pode ser categorizado em:

\subsubsubsection{BR/EPR}
%2.0 a 2.1

% REVER
Esta é a subdivisão mais popularizada do Bluetooth presente desde a versão $2.0$ do Bluetooth, onde 
as 
principais características são alta 
velocidade de transmissão alta em relação à outra categoria, baixo alcance e necessidade de conexão 
através de pareamento. O pareamento consiste em ..., onde os dispositivos confirmam a conexão. A 
partir disso, 
há um transmissão contínua de dados. Uma desvantagem é o consumo de energia considerável para o 
funcionamento do Bluetooth, devido a .... 
% taxa de dados
A taxa de transmissão gira em torno de 2Mb/s.
 
\subsubsubsection{BLE (Smart)}
% 4.0, 4.1, 4.2

% O que é
O \textit{Bluetooth Low Energy} (BLE) é a mais recente subdivisão do Bluetooth (Rever) incorporada 
a partir da versão 4.0 em 2011 além de ser a menos comum (Bluetooth-vs-BLE).
% Foco
BLE está centrado no baixo consumo de energia para permitir que certos 
dispositivos não precisem recarregar ou trocar suas fontes de carga, muitas vezes uma bateria, por 
longos períodos, que podem chegar a anos. 
% Pareamento
Para uma transmissão, ao contrário do BR/EPR, não é necessário um pareamento para realizar uma 
conexão, além disso esta tem curta duração, na ordem de milissegundos.
%Taxa e alcance
Além disso, a taxa de dados é baixa e o alcance alto. A baixa taxa de dados decorre do modo de 
funcionamento dos dispositivos BLE, aos quais, enviam dados em rajadas, ou seja, de tempos em 
tempos dados são transmitidos em forma de \textit{broadcast} e os dispositivos que estiverem 
conectados receberão esses dados. Nos intervalos de tempo em que o dispositivo não transmite, ele 
``dorme'', isto é, entra em modo de consumo mínimo a fim de poupar energia.

%Aplicação
A aplicação prática dessas características está na IoT através de \textit{beacons} e  
\textit{wearables}, aos quais incorporam o BLE. Os beacons foram introduzidos pela \textit{Apple} 
em XXXX com o nome de \textit{iBeacon}, com o objetivo de XXXXX. Com esses dispositivos é possível 
aprimorar a experiência do usuário em estabelecimentos como museus, supermercados, shoppings, 
estádios (referenciar aplicações), através da identificação de contexto, na qual, a partir da 
detecção de um beacon, uma aplicação móvel em um smartphone de um usuário pode exibir conteúdos, 
indicar promoções entre outros relacionados aquele dispositivo BLE.

\subsubsubsection{Dual-mode}
%
Esta categoria se refere a dispositivos, como \textit{smartphones} que precisam 
se conectar tanto com dispositivos BR/EDR, como fones de ouvido, e BLE, como 
\textit{beacons}(blu-core-specification).

\subsubsection{Bluetooth 5.0}

A versão 5.0 do Bluetooth foi lançada em seis de dezembro de 2016 (adopted-specfi) e trás consigo 
aprimoramentos em desempenho e segurança, garantindo duas vezes mais velocidade, quatro vezes mais 
alcance, oito vezes mais taxa de dados e, por fim, maior coexistência.  (specification-bluetooth5).

Com a nova versão, veio a flexibilidade para construção de soluções baseadas em necessidade. 
Parâmetros como alcance, velocidade e segurança podem ser regulados para diversos objetivos a 
depender das aplicações. (specification-bluetooth5).

Algumas atualizações contribuem para a redução de interferência com outras tecnologias sem fio, 
dessa forma, proporciona melhor coexistência entre dispositivos Bluetooth e de outras tecnologias, 
dentro do cenário emergente da IoT. (bluetooth-core-spec).




\subsection{RFID}

\subsection{NFC}

\subsection{Zigbee}

\subsection{Wi-Fi} % FAZER ??

\section{Aplicações}

Pouco tempo após à primeira referência a IoT, em 2000, uma empresa de grande porte, a LG, lançou 
um produto baseado na ideia de IoT: uma geladeira capaz de verificar se os produtos contidos nela 
foram reabastecidos \cite{survey-suresh}.



% TODO: PRESENTE
Atualmente, a IoT evoluiu e já está presente em diversos setores como 

% Smart home
% Smart grid
% Indústrias
% Transportes


% TODO: FUTURO


%%%%%%%%%%%%%%%%%%%%%%%%%%%%%%%%%%%%%%%%%%%%%%%%%%%%%%%%%%%%%%%%%
% ---
% Finaliza a parte no bookmark do PDF, para que se inicie o bookmark na raiz
% ---
\bookmarksetup{startatroot}% 
% ---

% ---
% Conclusão
% ---
\section*{Considerações finais}
\addcontentsline{toc}{section}{Considerações finais}





% ]  				% FIM DE ARTIGO EM DUAS COLUNAS
% ---

% ----------------------------------------------------------
% Referências bibliográficas
% ----------------------------------------------------------
\bibliography{ref}


\end{document}
